\section{Research Goals} \label{sec:intro/research_goals}

While the end goal for Osirion AS is to provide feedback which can be used to improve task performance, this thesis is merely a vital step along the way \textit{/means towards this end(?)}. In order to reliably produce feedback which is causally linked with divergence in performance, we need a systematic understanding of all aspects which may ultimately affect gameplay. Such aspects are inevitably both environmental and subject to circumstance. However, as the above section has eluded to, the significance of the cognitive load aspect should not be underestimated. Since emerging technologies allow for increasingly unobtrusive and commercially available eye-tracking, bridging the gap between such data and cognitive load classification may prove a significant value for performance analytics.

This raises the primary research goal for this thesis. It is to \textit{design and train a machine learning model to classify cognitive load aspects from patterns in eye-tracking data}. In this regard, any model that can perform reliable classification with accuracies beyond chance will be considered a success. Several model architectures will be researched and compared. Both fully end-to-end deep learning architectures and simpler models with elements of feature engineering, as well as a combination of the two. %Results will likely bring Osirion AS one step closer to per

Finally, to further substantiate results from the aforementioned goal, a secondary research goal will take a theoretical approach to the same problem. By consulting literature in neuroscience and cognitive psychology, the author will \textit{give a comprehensive presentation of cognitive load theory and subsequent correlations with ocular events}. Doing so will provide the tools with which an interesting conclusion can be made. It will also allow for more justified considerations in the design of model architecture, dataset and data recording environments.

% choice of model architecture during design, and considerations wh
% crucial features to include in the dataset, as well as allowing for more informed decisions when considering how data should be collected.

% merely a step on the way towards providing such feedback. 

% For this purpose, we need abundant insight into all aspects which may ultimately affect gameplay. This includes the physical setting; light intensity and temperature, posture and distance from screen. It includes prior history and trends of improvement. The holy grail, however, is likely linked to subjective perception.