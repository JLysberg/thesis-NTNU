\section{Motivation} \label{sec:intro/motivation}

\subsection{Eye-Tracking as an Emerging Technology} \label{sec:intro/motivation/eye_tracking}

Eye-tracking hardware has been available for a long time. Studies dating as far back as the 80s show the use of eye trackers together with computers, which achieved accuracies up to half a degree of visual angle \cite{colin1986}. Even before cameras and computers were advanced enough to measure anything accurately, mirrors have been used in reading exercises to observe gaze patterns and cognitive behavior \cite{vanGog2013}. The current state-of-the-art eye trackers provide massive accuracy improvements, higher sampling frequencies, better data quality, stability, and ease of use.

Most eye trackers in wide adoption today are used by researchers for research purposes, with data quality and price tags fit only for large research budgets. As an effect, manufacturers have had little incentive to promote the commercial availability of the hardware and its accompanying software. However, recent trends in the market have enabled the emergence of much more affordable eye-tracking. 
% Only recently have we seen emerging eye trackers with price tags below thousands of dollars.
This trend promotes its use even for the casual user, which has remodeled the commercial approach to eye-tracking applications.

Traditionally, the leading value proposition for eye-tracking has been as an assistive technology for people with disabilities, offering an improvement to the autonomy and quality of life for those in need of alternate input devices \cite{barry1994, corno2002}. More recently, the video-game industry has caught wind of the technology through a series of very promising studies over the past decades \cite{leyba2004, smith2006, tobii2017}. These studies suggest that eye-tracking hardware need not directly substitute existing control input devices but could instead serve to complement them. For example, game developers can make game graphics more immersive by letting the user's gaze point determine camera focus, depth of field, or light exposure. Game characters may interact differently with the user depending on whether they maintain eye contact. Eye-tracking provides a more challenging and immersive experience \cite{antunes2018}, and its adoption is only going to increase as the technology and its applications advance in the future.

\subsection{Eye-Tracking in \acrlong{esports}} \label{sec:intro/motivation/esports}

The ever-expanding competitive gaming environment drives another compelling use case for eye-tracking. As the video game industry is already worth more than the music and movie industries combined \cite{mangeloja2019}, all estimates show a positive trend in the interest for \acrfull{esports}. In fact, market reports show that the \acrshort{esports} audience reached 474 million people million in 2018, with a year-on-year growth of +8,7\% \cite{newzoo2021}. With this impressive growth comes the ever-increasing demand for competitive performance analytics.

There is always an incentive to be better at whatever game one plays, especially if the competitive scene is attractive. Naturally, the most effective method of improving performance is through direct feedback. Many amateur and pro players tend to subscribe to software services that provide targeted match feedback, as is evident by the success of companies such as Mobalytics and Shadow Esports. At Osirion AS, we aim to complement existing applications of match feedback with that which can be inferred from the analysis of eye-tracking data.

\subsection{Cognitive Load}

To reliably give targeted feedback to the user, we first need ways of distinguishing the good players from the great. Only then can we begin considering the aspects that separate them and help novice players reach higher levels of performance.

As the French poet Guillaume de Salluste so eloquently portrays them, our eyes can be considered "windows of the soul" \cite{hess1965} for their broad implications on cognition. As such, when eye-tracking is available as a data source, cognitive load is a natural step towards performance distinction. Tamara Van Gog, professor of educational sciences at the department of education at Utrecht University, states the following. "Eye tracking is not only a useful tool to study cognitive processes and cognitive load in computer-based learning environments but can also be used indirectly or directly in the design of components of such environments to enhance cognitive processes and foster learning." \cite{vanGog2013}. In a report, she refers to several studies where eye-tracking implementations have increased successful problem-solving.

It is safe to assume that a given task becomes decreasingly demanding with continued practice and increased experience. World-famous psychologist and Nobel Prize winner Daniel \textcite{kahneman2013} published a best-selling pop-science book in 2013, where he depicts two systems that drive the way we think. According to \textcite{kahneman2013}, the fast-thinking "System 1" is the most efficient actor when complex tasks are to be executed, as it is guided by intuition. This intuition serves to ease task execution such that capacity is freed from the slow-thinking conscious self. The catch, however, is that intuition needs to be trained. The field of psychology, as mentioned above, is commonly known as \acrfull{clt} and will be explained in detail in section \ref{sec:bt/CLT}. In short, cognitive load is a vastly complex metric that is subject to many confounding variables. Therefore, it is challenging to measure directly, and the development of accurate methods is an open problem.

As will become apparent in section \ref{sec:bt/cognitive_impacts}, ocular measures made available by modern eye-tracking have clear correlations with cognition. If a classification model could accurately predict levels of cognitive load from eye-tracking data, there is great potential for further research in user-targeted feedback in esports. Moreover, combining cognitive load with performance metrics may allow for the calculation of an index of cognitive capacity, mental efficiency, task expertise, or even intellect \cite{sweller1998}.

% If one were to accurately measure cognitive load in an environment where information and presentation can be controlled, there is reason to believe that individual differences in performance can be induced at later stages.

% Mental effort, memory workload, and other intrinsic processing demands are all factors which can be commonly attributed as cognitive load, and this is where the potential of modern eye-tracking equipment becomes apparent. Decades of research in psychology and neuroscience prove a significant correlation between these factors and various ocular events, such as pupil dilation and spontaneous eyeblink rate, just to name a few. 

% As Tamara Van Gog, professor of educational sciences at the department of education at Utrecht University states, "Eye tracking is not only a useful tool to study cognitive processes and cognitive load in computer-based learning environments but can also be used indirectly or directly in the design of components of such environments to enhance cognitive processes and foster learning." \cite{vanGog2013}. In a report, she refers to several studies where eye tracking implementations have led to an increase in successful problem solving. Such implementations are outside the scope of this thesis, however they serve as a motivational end to which further research may seek inspiration.