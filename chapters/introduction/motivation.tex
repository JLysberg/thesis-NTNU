\section{Motivation} \label{sec:intro/motivation}

\subsection{Eye-Tracking as an Emerging Technology}

Eye tracking hardware has been available for a very long time. Studies dating as far back as the 80s show the use of eye trackers in conjunction with computer hardware with accuracies up to about 1/2 degree of visual angle \cite{colin1986}. Even before cameras and computers were advanced enough to measure anything accurately, mirrors have been used in conjunction with reading excercises to classify saccades and fixations \cite{vanGog2013}. Current state of the art eye trackers provide massive improvements in accuracy, as well sampling frequency, data quality, data stability and overall ease of use.

Most eye trackers in wide adoption today, are used by researchers for research purposes, with data quality and price tags to match that of large research budgets. As an effect, manufacturers have had limited incentives to promote the commercial availability of the hardware and its accompanying software. Only recently have we seen emerging eye trackers with price tags below thousands of dollars. This enables their use for even the casual gamer, which in turn has remodelled the commercial approach to eye tracking applications.

Traditionally, the leading value proposition for eye tracking hardware has been as an assistive technology for people with disabilities, offering an improvement to the autonomy and quality of life for those in need of alternate input devices \cite{barry1994, corno2002}, but this trend is changing. More recently, the video-game industry has caught wind of the technology, through a series of very promising studies over the past decades \cite{leyba2004, smith2006, tobii2017}. What these studies suggest is that eye tracking hardware need not directly substitute existing control input devices, but could rather serve to complement them. For example, game developers can make game graphics more immersive by letting the user's gaze point determine camera focus, depth of field or light exposure. Game characters may interact differently with the user depending on whether the user is maintaining eye contact and so on. All in all, eye tracking provides a more challenging and immersive experience for the user \cite{antunes2018}, and it's adoption is only going to increase as the technology and its applications advances in the near future.

\subsection{Eye-Tracking in esports}

Another exciting use case for eye tracking is driven by the ever expanding competitive gaming environment. As the video game industry is already worth more than the music and movie industries combined \cite{mangeloja2019}, all estimates show a positive trend in the interest for \textit{Electronic Sports} (esports). In fact, market reports show that the esports audience reached 474 million people million in 2018, with a year-on-year growth of +8,7\% \cite{newzoo2021}. With this impressive growth comes the ever-increasing demand for competitive performance analytics.

There is always an incentive to be better at whatever game one is playing, especially if the competitive scene is attractive. Naturally the most effective method of improving performance is through direct feedback. Many amateur and pro players tend to subscribe to software services that provide targeted match feedback, as is evident by the success of companies such as Mobalytics and Shadow Esports. At Osirion AS, we aim to complement existing applications of match feedback with that which can be inferred from the analysis of eye-tracking data.

\subsection{Cognitive Load}

In order to reliably give targeted feedback to the user, we first need ways of distinguishing the good players from the great. Only then can we begin considering the aspects which separates them, and help novice players reach higher levels of performance.

Our eyes are can be considered "windows of the soul" \cite{hess1965}, for their broad implications on cognition. As such, when eye-tracking is available as a source of data, cognitive load is a natural step on the way towards performance distinction. As Tamara Van Gog, professor of educational sciences at the department of education at Utrecht University states, "Eye tracking is not only a useful tool to study cognitive processes and cognitive load in computer-based learning environments but can also be used indirectly or directly in the design of components of such environments to enhance cognitive processes and foster learning." \cite{vanGog2013}. In a report, she refers to several studies where eye tracking implementations have led to an increase in successful problem solving.

It is safe to assume that a given task becomes decreasingly demanding with continued practice and increased experience. As \textcite{kahneman2017} would put it in his 2017 pop-science best seller "Thinking, Fast and Slow", continued task exposure will gradually build a fast-thinking intuition. This intuition serves to ease task execution such that capacity is freed in the slow-thinking concious self. This field of psychology is commonly known as \acrfull{clt}, and will be explained in detail in section \ref{sec:bt/CLT}. However, as we will see, cognitive load is a necessarily complex metric that cannot be measured directly. 

% \acrfull{clt}, a field which will be explained in more detail in section \ref{sec:bt/CLT} is based upon three domains which define how such load is induced in the human brain. They are based on the complexity of presentation formats and the information itself, as well as the individual characteristics of the subject. It is these individual characteristics that are the holy grail of performance classification. 

As will become apparent in section \ref{sec:bt/cognitive_impacts}, ocular measures made available by modern eye-trackers have clear correlations with cognition. If a machine learning model were to accurately predict levels of cognitive load from eye-tracking data, there is great potential for further research in user-targeted feedback in esports. Moreover, combining cognitive load with performance metrics (such as match placement or rank) allows calculation of an index of cognitive capacity, mental efficiency, task expertise or even intellect \cite{sweller1998}.

% If one were to accurately measure cognitive load in an environment where information and presentation can be controlled, there is reason to believe that individual differences in performance can be induced at later stages.

% Mental effort, memory workload, and other intrinsic processing demands are all factors which can be commonly attributed as cognitive load, and this is where the potential of modern eye-tracking equipment becomes apparent. Decades of research in psychology and neuroscience prove a significant correlation between these factors and various ocular events, such as pupil dilation and spontaneous eyeblink rate, just to name a few. 

% As Tamara Van Gog, professor of educational sciences at the department of education at Utrecht University states, "Eye tracking is not only a useful tool to study cognitive processes and cognitive load in computer-based learning environments but can also be used indirectly or directly in the design of components of such environments to enhance cognitive processes and foster learning." \cite{vanGog2013}. In a report, she refers to several studies where eye tracking implementations have led to an increase in successful problem solving. Such implementations are outside the scope of this thesis, however they serve as a motivational end to which further research may seek inspiration.