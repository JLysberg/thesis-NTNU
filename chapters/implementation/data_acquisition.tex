\section{Data Acquisition}

Section \ref{sec:intro/research_goals} made the point that the end goal for this thesis is to classify cognitive load in humans from eye-tracking data. Unlike similar classification problems where data is readily available, a core challenge towards this goal lies in the aqcuisition of an eye-tracking dataset in an way that reliably encapsulates differing levels of cognitive load. For reasons detailed in section \ref{sec:bt/CLT}, the nature of \acrshort{clt} makes this a particularly difficult problem. 

The only measures of cognitive load which can truly be considered a "ground truth" in this regard, are subjective self-assessment questionnaires \cite{sharma2020, herbig2020}. Such measures, like the well-known \acrfull{nasa_tlx} \cite{hart1988}, makes remarkably accurate predictions of the cognitive load induced by a task though a long series of subjective questions and answers. Of course, the problem with such methods is their intrusiveness and low sampling frequency. There are less intrusive physiological measures available \cite{skulmowski2017}, such as \acrshort{eeg} and heart rate, but even these can only ultimately be a proxy that is more or less on par with pupillometry (see section \ref{sec:bt/cognitive_impacts}) alone.

As such, the most reasonable approach for the purposes of this thesis will be to manually record eye-tracking data in a setting where the task performed may be strictly controlled. Datasets can then be labelled by the difficulty and internal states of the task. In the light of considerations mentioned in section \ref{sec:bt/CLT}, a few assumptions need to be made for this approach to be viable:

\begin{enumerate}
    \item That task difficulty reliably reflects its load on working memory. \label{itm:Ass1}
    \item That the subject will always excert a mental effort to match the mental workload induced by the task. \label{itm:Ass2}
    \item That mental effort is a sufficient proxy for the more general cognitive load. \label{itm:Ass3}
    \item That mental effort remains fairly constant thoughout task execution and between two or more trials. \label{itm:Ass4}
\end{enumerate}

Of these assumptions, assumptions \ref{itm:Ass2} and \ref{itm:Ass4} may be the most uncertain. It is natural to believe that the subject's performance likely will increase with practice and decrease with fatigue. A given mental workload is known to induce reducing levels of mental effort with task experience \cite{tulga1980}. Fatigue may also affect task engagement, reducing the motivation to excert maximum mental effort on every trial. Since both subjective experience and fatigue are hard to control for, we can only accept that these limitations for what they are, and account for them when results are to be considered.