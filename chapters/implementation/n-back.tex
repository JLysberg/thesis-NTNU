\section{N-Back} \label{sec:impl/tasks}
\subsection{Overview}

Section \ref{sec:bt/cognitive_impacts} listed a number of standard experiments that are known to produce some form of mental strain on the subject. These experiments have been used extensively in many research fields, so their effects are constant and familiar. Therefore, they may be considered a sufficient proxy for cognitive load for this thesis. We will take insiration from REF and employ the N-Back task. It was chosen for its distinct levels of difficulty, which could be directly translated to levels of cognitive load for classification. Besides, its nature is simple, straightforward to implement, and requires nothing but a stimulus screen and an input device to deploy. 

The N-Back task operates by displaying a series of stimuli in the form of single letters on the screen, switching letters every other second. With each presentation, the subject is asked to indicate whether the stimulus matches the target by a keypress. Naturally, task difficulty increases with increasing N, demanding more and more information to be kept in working memory at any one time. The subject must constantly allocate attention to the task to perform well. Multiple studies have shown that both pupil diameter and \acrshort{ebr} increase with increasing N \cite{hopstaken2015, belayachi2015, brouwer2014, niezgoda2015}, proving its relevance for cognitive load classification. 

\subsection{Details}

For this particular use case, the author chose three difficulty levels; N=0, N=1, and N=2. Many related experiments also employ N=3. However, experience shows that most participants tend to find this level too difficult and give up early \cite{ayaz2007, izzetoglu2007}. Additionally, since this thesis merely aims to classify cognitive load, three levels would be more than sufficient.

The first level (N=0) is intended to impose no particular cognitive load on the subject besides requiring sustained attention. It is achieved by showing the target stimulus before the task commences. This way, working memory only needs to memorize one constant target. For levels N=1 and N=2, the target stimuli are dynamically determined by the stimulus N screens prior.

Experimental sessions are structured so that the subject is exposed to continuous engagement for only five minutes at a time. Every session consists of three N-Back blocks, each split into three block segments for every task difficulty (N=0, N=1, N=2). A value for N is picked randomly without replacement whenever a block segment is initiated. When all difficulties have been picked once, the next block begins. Block segments are structured as illustrated in figure \ref{fig:impl/NBackBlockSeg}. It begins with two milliseconds of segment presentation, where the subject is made aware of the segment difficulty. Then, nine series of screens are presented, each showing a stimulus for 500 milliseconds, followed by a 1500 milliseconds long blank screen. Finally, a fixation cross is presented for 6000 milliseconds. The idle, onset, execution, and offset labels in the figure will be detailed in section \ref{sec:impl/dataset/labeling}.

% The target stays constant throughout the task, such that the subject does not need to engage working memory besides , having a constant target throughout the trial.

\begin{figure}[h]
    \centering
    \includegraphics[width=\textwidth]{figures/impl_NBackBlock.png}
    \caption{Timeline for each N-Back block segment. The current example represents N=1. One N-Back block represents three such block segments, one for each difficulty level.}
    \label{fig:impl/NBackBlockSeg}
\end{figure}


