\section{N-Back and Cognitive Load}

The fact that the levels and states of the N-Back task are a sufficient proxy for cognitive load is a foundational pillar on which this entire thesis builds. If this were not the case, the classification models developed would have no value besides predicting various time intervals of a monotonous and uninteresting attention experiment. As detailed in section \ref{sec:impl/data_acq}, there are mainly five assumptions that need to hold for this pillar to stand. 

Since the author was also the subject in all trials, he may make some subjective assessments on assumptions \ref{itm:Ass1} and \ref{itm:Ass2}. First, assumption \ref{itm:Ass2} holds since the author was positively motivated to be actively engaged throughout the trials. Second, there can be no doubt that the task levels required distinctly different demands on working memory. Level 0 was trivial. After just a few stimulus screens, responding became entirely autonomous. Level 1 did require some sustained effort, and it was no longer possible to react correctly without actively engaging in the task. Level 2, however, was remarkably demanding. The subject required many consecutive trials to respond correctly to more than half of the stimuli. Therefore, we can confidently conclude that task levels sufficiently reflect the load on working memory. Despite this, it is worth noting that the load imposed was not necessarily linear with the difficulty level. While level 1 was slightly more challenging than level 0, level 2 was substantially more demanding than both the lower levels. This limitation is addressed by introducing the binary level label group, which only distinguishes between the least and most demanding levels. %Label groups are introduced in section \ref{sec:impl/dataset}.

Assumptions \ref{itm:Ass3} and \ref{itm:Ass5} are argued for in section \ref{sec:bt/CLT} and \ref{sec:bt/cognitive_impacts}, respectively. In short, the intensity of effort expended on a task (mental effort) is considered the essence of cognitive load \cite{hamilton1979} because other metrics are unreliable and subjective. Additionally, the effects of the \acrshort{pnr} and \acrshort{plr} may easily be accounted for by carefully controlling the recording environment. This includes the lighting and monitor position.

Finally, the potential uncertainties highlighted by assumption \ref{itm:Ass4} is visualized in figure \ref{fig:res/taskexposure}. For instance, we can see from the pupillary response of the top plot that data collected from the final quarter of task exposure deviates significantly from the prior three quarters. As discussed in section \ref{sec:bt/cognitive_impacts}, the pupil dilating and constricting fibers are tightly coupled with the central nervous system. Pupil dilation, in particular, is known to be triggered by the fight-or-flight response and the stimulus of the \acrshort{pns}. Therefore, what we see in the final few minutes of recording may just be attributed to impatience or fatigue. This result is unexpected given the experiment by \textcite{hopstaken2015}, reproduced in figure \ref{fig:bt/pupillary_expB}. They found the opposite relation; baseline pupil diameter decreased with time-on-task. The fact that our results conflict may indicate a lack of sufficient data quantity and quality. After all, only data from one subject in one session was recorded.

The eyeblink rate of the second plot in the same figure shows another interesting relationship. Here, the first quarter of recording indicates slightly elevated values compared to the rest of the session. As we also know from section \ref{sec:bt/cognitive_impacts}, \acrshort{ebr} is known to be affected by the assertion of cognitive control. The subject required a few consecutive trials of the hardest difficulty before responding correctly to most stimuli. This "learning phase" may have induced heightened levels of cognitive control, which may, in turn, have affected \acrshort{ebr}.