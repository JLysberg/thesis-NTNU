\section{Dataset Properties} \label{sec:disc/dataset}

Creating a processed dataset ready for analysis from a raw eye-tracking data stream turned out to be no trivial task. The high throughput of the 1200Hz eye-tracker made high-quality data abundant. However, data anomalies and noisy samples became just as problematic. Consequently, data processing was a significant part of the implementation process, entirely dwarfing other areas in terms of time invested. As detailed in section \ref{sec:impl/dataset}, subject blinking was the largest source of problems since the tracker continued recording empty data with closed eyes. Even after the gaze had been recovered following the blink event, hundreds of subsequent samples were contaminated by large amounts of noise. A representation of how the final dataset turned out is illustrated in figures \ref{fig:res/statelevel_pbr}, \ref{fig:res/statelevel_xy}, and \ref{fig:res/taskexposure}.

As the blue histograms above the three graphs of figures \ref{fig:res/statelevel_pbr} and \ref{fig:res/statelevel_xy} show, blinks seem to be remarkably consistent throughout every N-Back block segment. Blink occurrences are so consistent that the data anomalies they cause, as mentioned above, are apparent in visualizations. Pupil diameter in figure \ref{fig:res/statelevel_pbr}, as well as on-screen gaze on the y-axis of figure \ref{fig:res/statelevel_xy} was particularly affected by this. These graphs' sawtooth pattern suggests that a rapid data disturbance is caused after every blink occurrence, followed by a gradual rebound to pre-blink levels. 

Both the effect on gaze and pupil diameter may be explained by a limitation with the pupil- and corneal reflection method described in section \ref{sec:bt/ET_tech}. Since this algorithm relies on the outline of the pupil to determine both the pupil center and its diameter, it is natural to believe that occlusion of the pupil caused by closed eyelids will cause a disturbance. This effect is worth keeping in mind since a classification model is likely to pick up such features when making predictions. Since it is only caused by an anomaly and has no grounds in physiology, a model trained on such patterns is likely to generalize very poorly to unseen data.

% However, the fact that the pupil seems to constrict is more mysterious. The \acrshort{plr} would suggest the inverse relation since the dim lighting behind the eyelids likely would cause a dilating effect. A likely explanation is the same as for the observed gaze shift. 