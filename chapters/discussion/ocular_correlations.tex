\section{Ocular Correlations with Cognitive Load}

As the author spent several pages exploring in section \ref{sec:bt/cognitive_impacts}, the literature shows that there should be a clear correlation between the inner workings of the mind and various ocular events. With the dataset made available for this thesis, we can open the discussion on how these correlations play out in practice. 

\subsection{Pupillometry}

The pupillary response throughout each N-Back trial can be observed by the green graph in figure \ref{fig:res/statelevel_pbr}. First of all, values during the onset and offset task states show a striking resemblance with the results found by \textcite{kahneman1966}, reproduced in figure \ref{fig:bt/pupillary_expA}. They attributed pupil dilation on task onset to the increased number of digits that had to be kept in working memory at any one time. The pupil then constricted to pre-task levels after digits were unloaded. Similarly, the N-Back task requires the active engagement of working memory by memorizing a target stimulus. When transitioning from the idle state and into the first few stimulus screens, the subject is required to keep at least one letter memorized. Subsequently, when the subject was finally allowed to relax working memory by forgetting the target, the pupil again constricted to baseline levels.

% Even for level 0, the sustained attention necessary to detect a match in stimulus and target required consultation of working memory.

The fact that the effect is more pronounced for level 2 than level 0 further couples with the results by \textcite{kahneman1966}, which showed the same relation for long digit series as opposed to short. For one, figure \ref{fig:res/statelevel_pbr} show an average pupil diameter that is slightly higher for level 2. Even more evident, however, is its consistency. This consistency can be seen from the area drawn by the green graph's standard deviation. Especially during the execution state, level 2 pupil diameter shows consistent values above 0.7, with standard deviations closer to 0.1. Compared to average values of about 0.6 with standard deviations around 0.2 for level 0, the difference is significant. It seems clear from these results that phasic pupil dilation reflects working memory activation. Furthermore, the amount of information processed manifests itself in the magnitude of dilation.

% The value of the level 1 pupil diameter shows standard deviations of about 0.2, level 2 show closer to half this value.
% 

\subsection{\acrlong{ebr}} \label{sec:disc/ocular_correlations/ebr}

Both the blue histogram and the blue graphs of figure \ref{fig:res/statelevel_pbr} are an indicator of \acrshort{ebr}. The distinct groups of intervals in which blink events occur are of particular interest. The subject blinks about six times at a remarkably constant rate during the execution state. The groups are more discernible for level 2 than for level 0. Similar to the results found by \textcite{oh2012}, reproduced in figures \ref{fig:bt/oh2012A} and \ref{fig:bt/oh2012B}, this behavior may relate to the task-evoked \acrshort{ebr} detailed in section \ref{sec:bt/cognitive_impacts/ebr}. \textcite{oh2012} argues that distinct blinking behaviors can be observed at the onset of a task, which may be just what we observe in the present results.

With the results by \textcite{oh2012} in mind, one would expect to see an increase in baseline \acrshort{ebr} during task execution. Yet, what we see from figure \ref{fig:res/statelevel_pbr} is the opposite. Blink rate seems to decrease during task onset and increase during task offset. Perhaps this is the result of the N-Back task not inducing heightened levels of cognitive control as much as it induces cognitive load. As detailed in section \ref{sec:bt/cognitive_impacts}, different experiments are known to correlate with different physiological measures. If, for example, the Stroop task had been chosen instead, we might have seen a stronger correlation with \acrshort{ebr}.

However, the poor \acrshort{ebr} correlation with task states also shed some light on a limitation with the present setup of the N-Back task. In the experiment by \textcite{oh2012}, the rest condition lasted for as long as two minutes. Contrary to the present task, where the subject was only allowed six seconds of offset plus two seconds of rest, there is reason to believe that the task periods were too short to observe some of the inherent trends in the physiological responses. The pupillary data tell the same story. If the idle period was extended tenfold, the data quality and reliability would likely improve significantly.

\subsection{Gaze}

As explained in section \ref{sec:bt/cognitive_impacts/gaze}, gaze patterns do not have striking evidence in the literature which confirms their correlation with cognition. However, as we can see from figure \ref{fig:res/statelevel_xy}, there are features within the gaze coordinates recorded throughout task execution. First, the same sawtooth pattern in the pupillary data is quite apparent in the on-screen Y-coordinates. This observation is natural since the occlusion of the upper part of the pupil will disturb the pupil- and corneal reflection algorithm when calculating gaze position.

Another observation is the slight reduction in variance during task execution for both X- and Y-coordinates. Additionally, the variance reduction is most distinct in level 2 compared to 0. These results are consistent with what \textcite{williams1985} found in his study of a subject's field of view as a response to task-evoked cognitive load. Therefore, what we see here may be what he associates with a human tendency to subdue field of view when highly concentrated or under a heavy mental workload.