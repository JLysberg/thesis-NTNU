\chapter*{Preface}

This thesis is submitted as part of the master's degree requirements at the Department of Engineering Cybernetics at the Norwegian University of Science and Technology. The work presented has been carried out under the supervision of Assoc Prof. Sverre Hendseth at the Department of Engineering Cybernetics, NTNU.

% This master’s thesis is a continuation of a specialization project I conducted during the autumn of 20XX. As is customary, the specialization project is not published. This means that important background theory and methods from the project report will be restated in full throughout this report to provide the best reading experience. Below, a complete list of the material included from the
% specialization project is listed.

The thesis is, in some fashion, a continuation of a specialization project produced during the autumn of 2021. This project was not published, so any background theory originating from it will be restated in full throughout the text to maintain the reading experience. Specifically, this includes the following sections:
\begin{itemize}
	% \item Chapter 2 (Specifically sections 2.1, 2.2.2 and 2.2.3, with some changes to
	% section 2.2.3)
	% \item Chapter 3
    \item Chapter \ref{ch:intro}: Sections \ref{sec:intro/motivation/eye_tracking} and \ref{sec:intro/motivation/esports}.
    \item Chapter \ref{ch:bt}: Sections \ref{sec:bt/MLF} and \ref{sec:bt/ET_tech}.
\end{itemize}

% {\it It is important that you here also clearly describe which background material you have received; which information, software, equipment etc.~that have been made available to you, or form the basis for your work. Which help and support have you received, and from whom, during your work. For instance:}\\

During the project, the author was fortunate enough to get in touch with Prof. Mila Vulchanova and Prof. Giosuè Baggio from the Department of Language and Literature at Dragvoll. They manage the Language Acquisition and Language Processing Lab, which was kindly provided for the creation of an eye-tracking dataset. Furthermore, the Pytorch Deep Learning Library \cite{paszke2019} was used to streamline all machine learning development, training, evaluation, and testing.

% During the project, I have been provided multiple tools through NN who gave access to XX to be used in the work. The Matlab simulator used in this work/in Section 2 was developed by Ph.D. candidate XX, Department of Marine Technology, NTNU and extended to single-task end-effector control within the operational space by XX at the Department of Engineering Cybernetics, NTNU. During the master´s project I have further extended this simulator/adapted the simulator to the problem by...describe what you have done/your contributions to the simulator or other tools you have been given. XX and YY have also supported my work by answering questions regarding their work on modelling and control of robots which this thesis builds upon. An overview of the implementation of the ... method developed in this work is shown in Appendix A, Figur A.1. The "AA" and "BB" blocks, containing the ... and the ... algorithm, respectively, were received from PhD candidate NN. The "BB" block has been adapted in this work to include .... The rest of the guidance and control system has been implemented entirely during this work.

Unless otherwise stated, all figures and illustrations have been created by the author.

\section*{Acknowledgements}

First, I would like to thank my best friend and companion, Sophie Stokker, for being supportive and encouraging, despite my frequent silent treatments after a disappointing day of writing. Next, my fellow graduates deserve thanks for creating a work environment where writing a thesis has felt like a breeze. Finally, a special thanks go to Mila and Giosuè for taking me seriously and for showing genuine interest in what I was exploring. This project would never reach its conclusion if not for all the help, insight, and facilities provided. 

% \begin{itemize}
% 	\item I would like to thank
% \end{itemize}

% I would like to thank ... if you want to include any acknowledgments, this is the place to do it. \\[1cm]


\hfill {\it Jostein N. Lysberg}\\[-6mm]

\hfill {\it Trondheim, June 2022}