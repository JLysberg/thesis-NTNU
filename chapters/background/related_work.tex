\section{Related Work} \label{sec:bt/previous_work}

Performance analytics in gaming is not the only motivation for cognitive load classification. Many related studies have done similar work for different purposes. \textcite{heard2018} presents a review of 24 workload assessment algorithms, citing a use case for supervisory control environments such as the NASA control room for the Mars rover or a nuclear power plant. In such environments, one could imagine an adaptive workload system that dynamically monitors each employee's workload and changes autonomy or reassigns tasks accordingly. He argues that such systems could improve both workplace efficiency and employee health. \textcite{gerjets2014} presents a similar adaptive system for instructional design in digital environments. Instead of improving workplace efficiency, such a system would improve learning efficiency. Real-time assessment of cognitive workload could be used to actively adapt the presentation of instructions such that learning is optimized by maximizing germane cognitive load.

While some alternative algorithms were presented in the review by \textcite{heard2018}, the predominant method of cognitive load classification has been with machine learning models. A study by \textcite{hogervorst2014} compared the classification accuracies of various models on many combinations of data channels in a multimodal dataset. Data were recorded from the N-Back task, similar to the one described in section \ref{sec:impl/tasks}. They found that models trained on data from eye-tracking, EEG, and various physiological sensors achieved classification accuracies of over 90\%. Furthermore, models trained on eye-tracking data alone (pupillary, \acrshort{ebr}, and gaze) achieved accuracies up to 70\%. Similar results were found by \textcite{lobo2016} and \textcite{wilson2010}.

However, what limits cognitive load estimation in its current form is generalization across tasks and subjects. As concluded in the review by \textcite{heard2018}, training a model to distinguish cognitive load in more than one setting will severely limit its sensitivity since the impacts of cognition on physiology is complex and subjective. This is a tradeoff that must be assessed for every use case. 
% However, some generalization is likely necessary in all situations, 
% What all of these models lack, however, is generalizability across tasks and subjects. 

Generalization in cognitive load classification using eye-tracking was addressed in a dissertation by \textcite{appel2021}. He made a comprehensive effort to understand which features of eye-tracking data were most likely to generalize well and developed some methods which improved generalization accuracy. A result of particular interest was that pupil diameter and fixation frequency responded similarly for most participants in the training dataset. This suggests that the choice of data channels is more critical if generalization is a priority.