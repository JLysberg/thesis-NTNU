\section{Cognitive Impacts on the Ocular System} \label{sec:bt/cognitive_impacts}

As the reader might agree, vision is perhaps the most predominant sense of perception that humans enjoy. We can interpret vast amounts of information every second from the raw signals of millions of optic nerve fibers. This process is so complex that a significant fraction of the cortex is involved \cite{klatzky2012}. As such, it is natural to believe that our eyes are a good metric when studying the internal cognitive functions of our brain. In fact, the use of eye movements to study the inner workings of the mind have been exploited by cognitive psychologists for over two centuries \cite{wells1792}. 

This section will outline the most prominent impacts of cognition on the ocular system, where there are clear traces of correlation between mind and eyes. Some such correlations have clear causal links to neural circuitry and chemical release in the brainstem, while others are have merely been shown through empirical studies. 

% Cognitive psychologists have exploited this fact for over two centuries \cite{wells1792}, with studies in fields ranging from education and marketing to neuroscience and artificial intelligence.

% Some studies have even studied the possibility of inferring subject intention from their eye movements. \cite{ballard1992}, for instance, wanted to investigate whether gaze patterns were an indicator that preceded actions. To do this, he conducted an experiment where subjects were to move a set of colored blocks to match the pattern of a given model, while a mobile eye tracker was set up to record their eye movements. He found that there was indeed a strong link between a subject's eye movements and their actions. Gaze followed a clear pattern of checking out a block before picking it up and the model before placing it down. Similarly, \cite{land1999} measured a subject's eye movements as they performed daily tasks, such as brewing tea or making a sandwich, again finding that the subject's eyes always revealed their intentions before acting.

\subsection{Pupillometry}
\subsection{Spontaneous Eyeblink Rate}
\subsection{Eye Gaze}