\chapter*{Sammendrag}

Dette prosjektet utforsker dyp læring for klassifisering av kognitiv belastning ved å bruke et multimodalt øyesporingsdatasett. Arbeidet som presenteres er motivert av et fremvoksende marked for datadrevet ytelsesanalyse og helseovervåking innen spill.

Elektronisk Sport (e-sport) er en voksende bransje, og etter hvert som flere barn og voksne begynner å konkurrere, oppstår behovet for konkrete tilbakemeldi\-nger for å fremme prestasjon. De fleste gode tilbakemeldingene kommer imidlertid fra direkte coaching, som ikke er noe fornuftig alternativ for hvem som helst. Med fremveksten av kommersielt tilgjengelig øyefølging, presenterer løsningen seg selv som en automatisert ytelsesforskjellsmetode, som bruker okulære data for å utlede innsikt i kognisjon.

Løsningen er muliggjort av det okulære systemets enorme tilknytning til kognisjon. Spesielt kognitiv belastning har vist imponerende korrelasjoner med pupill\-estørrelse, spontan øyeblinkfrekvens og blikkmønstre. Et sett med toppmoderne modeller for dyp læring ble utforsket, utviklet og sammenlignet for klassifisering av tidsserier. Øyefølgingsdata ble tatt opp i et miljø med en nøye kontrollert oppgave for å pålitelig fange opp kognitiv belastning som målklasser i et treningsdatasett. Datasettet ble deretter merket etter oppgavetilstand og vanskelighetsgrad og brukt til å trene klassifikasjonsmodellene.

Datasettet viste korrelasjoner som var bemerkelsesverdig i samsvar med litteraturen. Selv om den var begrenset i mengde og generaliserbarhet, viste den funksjoner som var distinkte nok til å trene nevrale nettverk for innad-oppgave og innad-bruker klassifisering. Til slutt kunne de beste modellene skille mellom fire tilstander av oppgaveeksponering med en nøyaktighet på 71\% og tre nivåer av kognitiv belastning med en nøyaktighet på 61\%. Disse resultatene beviser at konseptet fungerer. De legger grunnlaget for videre forskning på ikke-påtrengende midler for prestasjonsanalyse. Fremtidig arbeid bør ta for seg datasettdesign og generaliserbarhet for å generere modeller som pålitelig kan skille kognitiv belastning mellom brukere og på enhver oppgave.