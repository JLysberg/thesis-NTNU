
% From https://www.overleaf.com/learn/latex/Glossaries

% \makeglossaries % Prepare for adding glossary entries


% \newglossaryentry{latex}
% {
%         name=latex,
%         description={Is a mark up language specially suited for
% scientific documents}
% }

% \newglossaryentry{bibliography}
% {
%         name=bibliography,
%         plural=bibliographies,
%         description={A list of the books referred to in a scholarly work,
% typically printed as an appendix}
% }

% \newglossaryentry{maths}
% {
%     name=mathematics,
%     description={Mathematics is what mathematicians do}
% }


% --------------------
% ----- Acronyms -----
% --------------------



% \newacronym{phd}{PhD}{philosophiae doctor}
% \newacronym{CoPCSE}{CoPCSE@NTNU}{Community of Practice in Computer ScienceEducation at NTNU}
% \newacronym{gcd}{GCD}{Greatest Common Divisor}
\newacronym{clt}{CLT}{Cognitive Load Theory}
\newacronym{cnn}{CNN}{Convolutional Neural Network}
\newacronym{rnn}{RNN}{Recurrent Neural Network}
\newacronym{plr}{PLR}{Pupil Light Response}
\newacronym{pnr}{PNR}{Pupil Near Response}
\newacronym{ppr}{PPR}{Psychosensory Pupil Response}
\newacronym{sns}{SNS}{Sympathetic Nervous System}
\newacronym{pns}{PNS}{Parasympathetic Nervous System}
\newacronym{lc}{LC}{Locus Coeruleus}
\newacronym{ne}{NE}{Noradrenaline}
\newacronym{da}{DA}{Dopamine}
\newacronym{ebr}{EBR}{Spontaneous Eyeblink Rate}

\printglossary[type=\acronymtype] % Print acronyms